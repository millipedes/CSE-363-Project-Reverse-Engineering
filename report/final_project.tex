\documentclass{report}

\usepackage{algorithm}
\usepackage{algpseudocode}
\usepackage{amsmath}
\usepackage{amsfonts}
\usepackage{amssymb}
\usepackage{enumitem}
\usepackage{fancyhdr}
\usepackage{graphicx}
\usepackage{mathrsfs}
\usepackage{sectsty}
\usepackage{soul}

\newcommand{\n}{$\char`\\ n$}
\newcommand{\tab}{$\quad$}
\newcommand{\smalls}{\par\smallskip}
\newcommand{\meds}{\par\medskip}
\newcommand{\HRule}{\rule{\textwidth}{1pt}}
\newcommand{\biigs}{\par\bigskip}
\newcommand{\csfont}[1]{\fontfamily{cmtt}\selectfont #1}
\newcommand{\XOR}{\mathbin{\char`\^}}
\newcommand{\smalltitle}[1]{\textit{#1}}

\pagestyle{fancy}
\fancyhf{}
\rhead{\thepage}
\lhead{\textit{CSE 363: Final Project: Reverse Engineering}}
\rfoot{}

\begin{document}
\begin{titlepage}
    \begin{center}
        \vspace{0.25cm}
        \hrule
        \vspace{0.1cm}
        \hrule
        \vspace{0.25cm}
        {\LARGE{\textbf{CSE 363}}}
        \par
        \vspace{0.25cm}
        {\large{\textbf{Final Project: Reverse Engineering}}}
        \par
        \vspace{0.25cm}
        {\large{\textbf{New Mexico Institue of Mining and Technology}}}
        \par
        \vspace{0.25cm}
        \textbf{Matthew C. Lindeman, Luke Shankin, Nikki Sparacino, Cameron
        Savage}
        \par
        \vspace{0.25cm}
        {\large{\textbf{Date \today}}}
        \vspace{0.25cm}
        \hrule
        \vspace{0.1cm}
        \hrule
        \vfill
    \end{center}
\end{titlepage}

\section*{1 Heap's Algorithm}
  \begin{tabular}{c|l}
    \hline
    1 & {\csfont{ulong dbg.main(ulong argc, int64$\_$t argv)}} \\
    2 & {\csfont{}} \\
    3 & {\csfont{$\{$}} \\
    4 & {\csfont{\tab \tab int32$\_$t iVar1;}} \\
    5 & {\csfont{\tab \tab int64$\_$t arg1;}} \\
    6 & {\csfont{\tab \tab ulong str;}} \\
    7 & {\csfont{\tab \tab ulong var$\_$24h;}} \\
    8 & {\csfont{\tab \tab ulong arr;}} \\
    9 & {\csfont{\tab \tab uint32$\_$t i;}} \\
    10 & {\csfont{\tab \tab int32$\_$t var$\_$8h;}} \\
    11 & {\csfont{\tab \tab ulong val;}} \\
    12 & {\csfont{\tab \tab }} \\
    13 & {\csfont{\tab \tab // int main(int argc,char ** argv);}} \\
    14 & {\csfont{\tab \tab val.$\_$0$\_$4$\_$ = 7;}} \\
    15 & {\csfont{\tab \tab if (argc == 2) $\{$}} \\
    16 & {\csfont{\tab \tab \tab \tab val.$\_$0$\_$4$\_$ = sym.imp.atoi(*(argv + 8));}} \\
    17 & {\csfont{\tab \tab $\}$}} \\
    18 & {\csfont{\tab \tab arg1 = sym.imp.calloc();}} \\
    19 & {\csfont{\tab \tab for (var$\_$8h = 1; var$\_$8h $<$= val; var$\_$8h = var$\_$8h + 1) $\{$}} \\
    20 & {\csfont{\tab \tab \tab \tab *(var$\_$8h * 4 + -4 + arg1) = var$\_$8h;}} \\
    21 & {\csfont{\tab \tab $\}$}} \\
    22 & {\csfont{\tab \tab iVar1 = dbg.fact(val);}} \\
    23 & {\csfont{\tab \tab for (i = 0; i $<$ iVar1; i = i + 1) $\{$}} \\
    24 & {\csfont{\tab \tab \tab \tab dbg.heaps(arg1, val, i);}} \\
    25 & {\csfont{\tab \tab \tab \tab sym.imp.printf("$\%$d\t|");}} \\
    26 & {\csfont{\tab \tab \tab \tab dbg.printarr(arg1, val);}} \\
    27 & {\csfont{\tab \tab \tab \tab if (i $\%$ 6 == 5) $\{$}} \\
    28 & {\csfont{\tab \tab \tab \tab \tab \tab sym.imp.putchar(10);}} \\
    29 & {\csfont{\tab \tab \tab \tab $\}$}} \\
    30 & {\csfont{\tab \tab \tab \tab if (i + ((i / 6 + (i $>$$>$ 0x1f) $>$$>$ 2) - (i $>$$>$ 0x1f)) * -0x18 == 0x17) $\{$}} \\
    31 & {\csfont{\tab \tab \tab \tab \tab \tab sym.imp.puts(0x402009);}} \\
    32 & {\csfont{\tab \tab \tab \tab $\}$}} \\
    33 & {\csfont{\tab \tab $\}$}} \\
    34 & {\csfont{\tab \tab sym.imp.free(arg1);}} \\
    35 & {\csfont{\tab \tab return 0;}} \\
    36 & {\csfont{$\}$}} \\
    \hline
  \end{tabular}
  \biigs
  Some insightful analysis here.
  \biigs
  \begin{tabular}{c|l}
    \hline
    1 & {\csfont{void dbg.swap(uint *arg1, uint *arg2)}} \\
    2 & {\csfont{}} \\
    3 & {\csfont{$\{$}} \\
    4 & {\csfont{\tab \tab uint uVar1;}} \\
    5 & {\csfont{\tab \tab ulong b;}} \\
    6 & {\csfont{\tab \tab ulong a;}} \\
    7 & {\csfont{\tab \tab ulong tmp;}} \\
    8 & {\csfont{\tab \tab }} \\
    9 & {\csfont{\tab \tab // void swap(int * a,int * b);}} \\
    10 & {\csfont{\tab \tab uVar1 = *arg1;}} \\
    11 & {\csfont{\tab \tab *arg1 = *arg2;}} \\
    12 & {\csfont{\tab \tab *arg2 = uVar1;}} \\
    13 & {\csfont{\tab \tab return;}} \\
    14 & {\csfont{$\}$}} \\
    \hline
  \end{tabular}
  \biigs
  Some insightful analysis here.
  \biigs
  \begin{tabular}{c|l}
    \hline
    1 & {\csfont{void dbg.printarr(int64$\_$t arg1, ulong arg2)}} \\
    2 & {\csfont{}} \\
    3 & {\csfont{$\{$}} \\
    4 & {\csfont{\tab \tab ulong arr;}} \\
    5 & {\csfont{\tab \tab ulong i;}} \\
    6 & {\csfont{\tab \tab }} \\
    7 & {\csfont{\tab \tab // void printarr(int * arr,int length);}} \\
    8 & {\csfont{\tab \tab for (i.$\_$0$\_$4$\_$ = 0; i $<$ arg2; i.$\_$0$\_$4$\_$ = i + 1) $\{$}} \\
    9 & {\csfont{\tab \tab \tab \tab sym.imp.printf(0x40200b, *(arg1 + i * 4));}} \\
    10 & {\csfont{\tab \tab $\}$}} \\
    11 & {\csfont{\tab \tab sym.imp.putchar(10);}} \\
    12 & {\csfont{\tab \tab return;}} \\
    13 & {\csfont{$\}$}} \\
    \hline
  \end{tabular}
  \biigs
  Some insightful analysis here.
  \biigs
  \begin{tabular}{c|l}
    \hline
    1 & {\csfont{int32$\_$t dbg.fact(ulong arg1)}} \\
    2 & {\csfont{}} \\
    3 & {\csfont{$\{$}} \\
    4 & {\csfont{\tab \tab int32$\_$t iVar1;}} \\
    5 & {\csfont{\tab \tab ulong x;}} \\
    6 & {\csfont{\tab \tab }} \\
    7 & {\csfont{\tab \tab // int fact(int x);}} \\
    8 & {\csfont{\tab \tab if (arg1 $<$ 1) $\{$}} \\
    9 & {\csfont{\tab \tab \tab \tab iVar1 = 1;}} \\
    10 & {\csfont{\tab \tab $\}$}} \\
    11 & {\csfont{\tab \tab else $\{$}} \\
    12 & {\csfont{\tab \tab \tab \tab iVar1 = dbg.fact(arg1 - 1);}} \\
    13 & {\csfont{\tab \tab \tab \tab iVar1 = iVar1 * arg1;}} \\
    14 & {\csfont{\tab \tab $\}$}} \\
    15 & {\csfont{\tab \tab return iVar1;}} \\
    16 & {\csfont{$\}$}} \\
    \hline
  \end{tabular}
  \biigs
  Some insightful analysis here.

\end{document}
